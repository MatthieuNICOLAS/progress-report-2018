\documentclass[11pt]{article}

\usepackage[utf8]{inputenc}
\usepackage{a4wide}

\usepackage{acronym} % \ac[p], \acl[p], \acs[p], \acf[p]

% Acronyms
% --------
\acrodef{CRDT}[CRDT]{Conflict-free Replicated Data Type}
\acrodefplural{CRDT}[CRDTs]{Conflict-free Replicated Data Types}
\acrodef{Loria}{Lorraine Research Laboratory in Computer Science and its Applications}
\acrodef{OT}{Operational Transformation}

% This document has to be filled (in English of French) by each PhD
% student (and its supervisor for the second part), except for those
% who will defend their thesis within the next weeks.
% Once filled and signed, this document has to be sent to the CMI
% (francoise.laurent@loria.fr) and to the members of the Comite de
% Suivi de Thèse (formerly called referents).
% Deadline: May 15, 2018.

\begin{document}
\bigskip
\centerline{\Large\textbf{Progress report of a PhD Thesis -- 2017-2018}}
\bigskip
\bigskip

\noindent\textbf{PhD student:}
Matthieu NICOLAS
\\

\noindent\textbf{PhD year:}
1\textsuperscript{st} year
\\

\noindent\textbf{Supervisors:}
Olivier PERRIN and Gérald OSTER
\\

\noindent\textbf{Laboratory:}
\ac{Loria}
\\

\noindent\textbf{Funding of the thesis:}
Doctoral contract
\\

\noindent\textbf{Subject of the PhD thesis:}
Efficient (re)naming in \acp{CRDT}
\\


\section*{Short description of the subject}
% general context, considered approach, scientific objectives
% compared to the state of the art

\hspace{1em} In order to serve an ever-growing number of users and provide an increasing volume of data,
large scale systems such as data stores or collaborative editing tools have to adopt a distributed architecture.
However, as stated by the CAP theorem, such systems cannot ensure both strong consistency and high availability in case of network partitions.
As a result, literature and companies increasingly adopt the optimistic replication model known as eventual consistency to replicate data among nodes.
This consistency model allows replicas to temporarily diverge to be able to ensure high availability, even in case of network partition.
Each node owns a copy of the data and can edit it, before propagating updates to others.
A conflict resolution mechanism is however required to handle updates generated in parallel on different replicas.

An approach, which gains in popularity since a few years, proposes to define \acfp{CRDT}.
These data structures behave as traditional ones, like the \emph{Set} or \emph{Sequence} data structures, but are designed for a distributed usage.
Their specification ensures that concurrent updates are resolved deterministically, without requiring any kind of agreement, and that replicas eventually converge immediately after observing some set of updates,
thus achieving \emph{Strong Eventual Consistency}.

% TODO: Rework the following paragraph
To achieve convergence, \acp{CRDT} proposed in the literature mostly rely on unique identifiers to reference updated elements.
To generate such element identifiers, nodes often use their own identifier as well as a logical clock.
Thus, regarding to how node identifiers are generated, the size of element identifiers usually increases with the number of nodes.
Furthermore, element identifiers have to comply to additional constraints according to the \ac{CRDT}, for example forming a dense set in case of a sequence data structure.
In this case, element identifiers' size also increases according to the number of elements contained in the data structure.
Therefore, the size of element identifiers is usually not bounded.

Since the size of identifiers attached to each element is not bounded, the overhead of the replicated data structure increases over time.
Since nodes have to store and broadcast the identifiers, the application's performances and efficiency decrease over time.
This impedes the adoption of \acp{CRDT}.

This PhD aims to address this issue.
A first approach is to study identifiers proposed in the literature to list existing constraints on
identifiers and their consequences on identifiers generation
in order to propose more efficient specifications of identifiers.
A second approach is to study this issue as a particular case of the renaming problem
and to propose mechanisms to rename identifiers in order to reduce their size,
still without requiring any kind of agreement between nodes.

\section*{Main results}
% studies and works achieved, results obtained with respect to the
% objectives of the thesis; difficulties encountered

\hspace{1em} During this first year, I focused on designing a renaming mechanism for \emph{Identifier-based Sequence \ac{CRDT}}.

\subsection*{Identifier-based Sequence \acp{CRDT}}

\hspace{1em} \emph{Identifier-based Sequence \acp{CRDT}} are data structures used to represent replicable lists.
Two operations are defined to update them : \emph{insert} and \emph{delete}.

The data structures attach an identifier to each inserted element.
These identifiers allow us to achieve transaction-less and commutative updates by uniquely identifying each element and ordering them.

The downside of this approach is the increasing size of the identifiers.
Since the identifiers are used to order the elements, they have to form a dense set so that users are always able to insert a new element between two others.
However, two identifiers of the same size can be contiguous.
When inserting a new element between two such identifiers, we have no other choice than to increases the size of the generated identifier to be able to generate one respecting the intended order.

\subsection*{Renaming mechanism}

% Rename identifiers to have different but shorter ones

% Other sites can perform updates concurrently

% Cannot perform these updates on a renamed site directly
% - Would not insert the element at the correct position
% - Would not delete the correct element

% Have to use OT
% - Define transformation functions to transform the concurrent operations before applying them
% - Need to transform insertions and deletions according to the renaming
% - Need to transform the renaming according to insertions and deletions

\begin{itemize}
    \item Designed a renaming mechanism which does not require any kind of coordination
    \begin{itemize}
        \item Performing a local renaming to reduce the overhead of the data structure is straightforward
        \item Broadcasting the renaming to other nodes also
        \item The issue is to deal with concurrent updates to a renaming (for example, deleting an element using its identifier)
        \item The main idea is to transform concurrent updates to a renaming to preserve the intention of the user
    \end{itemize}
    \item Introduced an optimization to reduce the bandwith comsuption of the renaming operation
    \item Implementing its prototype in \emph{MUTE}, the collaborative editing tool from the team
\end{itemize}

\section*{Plan for next year}
% remaining problems to be considered, approach; precise planning (in
% particular for those who are at least in 3rd year, for the report
% writing and the defense)

% Need to evaluate the contribution
\hspace{1em} The plan for the next year is to complete the implementation of the proposed renaming mechanism
and to proceed to its validation.

% - Simulation approach
%   - Run simulations to benchmark performances
%   - Compare results to state-of-the-art algorithms
%   - Require logs

To validate the mechanism, the first approach is to run simulations to benchmark its performances.
The goal is to measure the gain in memory usage provided by the mechanism
as well as its cost on the performances of the application.
% TODO: Insert sentence about illustrating that the mechanism is worth it
These simulations could also be used to compare the efficiency of different strategies
regarding the conditions triggering the renaming process.

The main issue regarding this approach is the authenticity of such simulations.
Indeed, to run truthful simulations, the behaviour of the agents representing the collaborators should be credible.
It is then required to study the literature on the behaviour of users in collaborations to reproduce their actions faithfully.

% - Verification approach

The second approach is to prove that transformation functions proposed are correct.
Literature on \ac{OT} provides several constraints that such functions have to respect
to ensure the correctness of the system.
% TODO: Insert sentence about determining which constraints to comply with and proving that we do

\section*{Publications}
% if any
\hspace{1em} No publications currently.

\section*{Project after the thesis}
% for 3rd year (and beyond) PhD students: professionnal project
\hspace{1em} My current project is to pursue an academic career as a lecturer.

\section*{Scientific and professional modules validated}
% list of all modules validated from the beginning of the thesis
% Remark: do not forget to add them in Adum, together with your publications

\subsection*{Scientific modules}

\begin{itemize}
    \item Réplication et cohérence des données
\end{itemize}

\subsection*{Professionnal modules}

\begin{itemize}
    \item Fi4 152 E Sauveteur Secouriste du Travail (SST)
    \item Fi4 162 C Formation à la communication orale et corporelle en milieu professionnel
    \item Fi4 282 Outils numériques pour la pédagogie (plateforme Arche, studio professeur)
\end{itemize}

\bigskip
\bigskip
\noindent\textbf{Date and signature of the PhD student}
............

%%%%%%%%%%%%%%%%%%%%%%%%%%%%%%%%%%%%%%%%%%%%%%%%%%%%%%%%%%%%%%%%%%
\newpage

\section*{Opinion of the supervisor}

\noindent\textbf{Name of the supervisor:}
............
\\

\noindent\textbf{Opinion on this progress report:}
............
\\

\noindent\textbf{Agreement for an additional year?}
Yes/No
% if No, please justify
\\

\noindent\textbf{Date of the defense:}
% this can be very approximative
............
\\

\bigskip
\bigskip
\noindent\textbf{Date and signature of the supervisor}
............


\end{document}
